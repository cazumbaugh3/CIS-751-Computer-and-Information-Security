\documentclass[12pt]{article}
\author{Chuck Zumbaugh}
\title{CIS 751 Lecture Assignment 5}

\usepackage{listings}

\begin{document}
\maketitle

Let's assume we have a stack that looks something like below:
\bigbreak
\begin{tabular}{|c|}
\hline
Lower Addresses \\
\hline
EBP \\
\hline
EIP \\
\hline
Format string address \\
\hline
$addr$ \\
\hline
Rest of input \\
\hline
Higher addresses \\
\hline
\end{tabular}
\bigbreak

And also we have a program with the following:

\begin{lstlisting}[language=c]
int i = 0; // Some variable used in printf()
printf(input, &i); // Input is some user created data
\end{lstlisting}

We could then overwrite $i$, say with 11, using the string $"\textbf{Hello world\%n}"$. When executed, printf() will write the number of bytes printed (11 in this case) to the location pointed to by $\&i$. However, we can also overwrite arbitrary stack locations provided we know the address we want to write to. When printf() encounters a format specifier it will call va\_arg() and return the argument pointed to by va\_list. Thus, we can specify an address to write to, and a number of format specifiers to move the va\_list pointer to this address. We would need to move the pointer from somewhere in \textit{Rest of input} to $addr$ using input specifiers (ex \%x). Then, when \%n is encountered, it will write the number of bytes in printf() to that address.

\end{document}
