\documentclass[12pt]{article}
\title{CIS 751 Project Proposal}
\author{Chuck Zumbaugh}

\begin{document}
\maketitle

\noindent
\textbf{Proposed Title:} Memory corruption vulnerabilities: Modern safeguards and their shortcomings

\bigbreak
\noindent\textbf{Overview:} Memory corruption vulnerabilities are common and represent a significant portion security bugs. They are relatively easy to introduce, particularly in memory unsafe languages such as C/C++, and allow for arbitrary code execution when exploited. Significant work has been done to improve the security of modern systems through hardware and OS-level protections. Nevertheless, these bugs remain exploitable and many zero-days are the result of memory corruption. The objective of this review will be to explore the following strategies to protect against memory corruption vulnerabilities, including how they function, the problem they solve, and current methods attackers can use to circumvent them.  

\begin{enumerate}
\item Overview of memory corruption vulnerabilities
\item Modern safeguards and strategies to circumvent them
\begin{enumerate}
\item Address space layout randomization (ASLR)
\item Executable-space protection
\item Pointer authentication codes
\item Memory tagging extension
\end{enumerate}
\end{enumerate}

\end{document}
