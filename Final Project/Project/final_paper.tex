\documentclass[conference]{IEEEtran}
% \IEEEoverridecommandlockouts
% The preceding line is only needed to identify funding in the first footnote. If that is unneeded, please comment it out.
\usepackage{cite}
\usepackage{url}
\usepackage{amsmath,amssymb,amsfonts}
\usepackage{algorithmic}
\usepackage{tikz}
\usetikzlibrary{positioning}
\usepackage{graphicx}
% LaTeX Symbols: https://tug.ctan.org/info/symbols/comprehensive/symbols-a4.pdf
\usepackage{halloweenmath}
\usepackage{caption}
\captionsetup{justification=raggedright,singlelinecheck=false} % Align figure captions to the left
\usepackage{textcomp}
\usepackage{xcolor}
\def\BibTeX{{\rm B\kern-.05em{\sc i\kern-.025em b}\kern-.08em
    T\kern-.1667em\lower.7ex\hbox{E}\kern-.125emX}}
        
\begin{document}

\title{Securing the software supply chain}

\author{\IEEEauthorblockN{Charles Zumbaugh}
\IEEEauthorblockA{\textit{Dept. of Computer Science} \\
\textit{Kansas State University}\\
Manhattan, Kansas \\
cazumbaugh@ksu.edu}
}

\maketitle

\begin{abstract}
Modern software development depends heavily on layers of abstraction 
such as libraries, frameworks, cloud infrastructure, build tooling, and other 
software built and maintained by others. This speeds innovation and makes the 
development of complex systems more accessible by allowing developers to build 
upon previous work. Given it's importance to modern development, the software 
supply chain represents an important attack vector. Attackers are increasingly 
targeting this ecosystem to mount attacks, and recent incidents highlight the scale 
of the fallout when these attacks are successful. This review discusses the primary 
attack vectors in the software supply chain, security measures and frameworks to 
defend against these attacks, and recent high-profile incidents.
\end{abstract}

\begin{IEEEkeywords}
memory corruption, security, vulnerability, systems, malware
\end{IEEEkeywords}

\section{Introduction}
As computers become more embedded in everyday life, the security of these systems have become a critical concern. Cyber attacks against government and non-government organizations have increased in recent years, particularly against healthcare and large, multinational companies \cite{hammouchi2019digging}. As personal data is increasingly being collected and stored on servers, the scale of these breaches is also increasing. For example, a recent breach of Change Healthcare in 2025 exposed the personal data of more than 192 million individuals in the United States \cite{privacyrights2025}. The financial implications of security vulnerabilities are substantial, and IBM estimated the cost of a data breach in 2025 to be \$4.4 million \cite{ibm2025}.

Modern software is built upon layers of reusable abstractions such as libraries, frameworks, cloud infrastructure, and build tools that are built and maintained by third parties. This ecosystem is known as the software supply chain and it involves numerous, globally-distributed participants building software used at various phases in the development process. The 2025 Open Source Security and Risk Analysis report found that 97\% of evaluated codebases contained open source software, with 100\% of codebases in the EdTech, internet, and mobile app sectors containing open source software \cite{blackduck2025}. 

Given the massive scale of open source software and the software supply chain in general, it is not surprising that attackers are increasingly targeting this ecosystem. Sonatype reported that 34,319 new open source malware packages were identified in quarter 3 of 2025, representing a 140\% increase from quarter 2 2025 \cite{sonatype2025}. Recent attacks such as Log4j and SolarWinds highlight the importance of securing this ecosystem and the consequences of a successful attack. Aside from the economic damages associated with these attacks, distrust in the software ecosystem will undoubtedly slow technological innovation. Thus, it is essential to develop tools and frameworks to guard against malicious actors.

\section{Software Supply Chain Attack vectors}
Software supply chain attacks are considered to have three major attack vectors: dependencies, build infrastructure, and humans \cite{williams2025research}. These attack vectors span the entire software lifecycle, and thus, security risks exist in each phase of this lifecycle. This review will focus primarily on software dependencies and artifact registries, cloud infrastructure, and build systems.

\subsection{Dependencies}
Vulnerabilities and malware included in open source and third party dependencies represent a critical security threat. In many cases, unintentional vulnerabilities are included in these dependencies. While such vulnerabilities are generally discovered, patched, and made public, developers and administrators may be slow to patch. For example, many systems were still vulnerable to attacks such as Heartbleed and Shellshock after the patch was released, in some cases years after \cite{10154229}. A recent analysis of major package managers found that technical lag is common, with the majority of fixed version declarations being outdated and a significant number of flexible version declarations being outdated \cite{9359290}. While there are likely many reasons for this, research has shown a strong presence of technical lag caused by the use of dependency constraints in the JavaScript package manager, npm, suggesting that developers may be reluctant to update dependencies due to backwards compatibility \cite{zerouali2018empirical}. This idea is supported by qualitative research which demonstrated that developers are more likely to adopt a security fix \textit{not} bundled with functionality improvements because they are less likely to introduce breaking changes \cite{pashchenko2020qualitative}. Maintaining software ecosystems is costly, especially if updates are frequent \cite{berhe2023maintenance}, but the technical lag generated by missing updates can result in an increased risk of critical vulnerabilities.

Attackers may also attempt to infect programs by typosquatting, the practice of uploading a package with a similar name as that of a very popular package (ex. reacte vs react). If a developer makes a typo while installing the package (ex. \textit{npm install reacte}), the malicious software is installed, which can target either the developer's or end user's machine. This practice is common in software libraries and other areas of the internet. While the effectiveness at tricking users varies based on how the fake name is composed. Users are more likely to correctly identify typosquatting that adds characters to the name, and are more likely to be deceived by permutations and substitutions of characters \cite{spaulding2017understanding}. Due to its prevalence, several software tools have been developed in recent years to guard against these attacks. A recently developed tool for identifying and reporting typosquatted imports, TypoGard, was able to flag up to 99.4\% of known typosquatted imports on npm, PyPI, and RubyGems \cite{taylor2020defending}. At a high level TypoGard compares the package name against a list of popular packages, and flags it if the name matches after transformations. These tools also exist to protect against typosquatted URLs online with good results. For example, TypoWriter uses a recurrent neural network to identify probable typo variations during the pre-registration phase \cite{ahmad2019typowriter}, and TypoAlert is a Chrome extension that can alert users to alert users when they attempt to visit a typosquatted website \cite{blefari2024typoalert}.

\subsection{Artifact Registries}
An artifact registry or repository is a centralized system that stores, manages, and versions software artifacts such as libraries, compiled code, and container images. These registries provide a mechanism for package developers to deploy and maintain their software, and allow developers to easily consume these packages using package managers. Attacks on artifact registries are closely related to dependency vulnerabilities, as they commonly rely on developers unknowingly installing malicious software. Attackers attempt to hijack legitimate packages in these registries by attacking deleted packages, compromising the maintainer's account, or by exploiting package relocation in decentralized registries \cite{gu2023investigating}. In the first case, attackers simply monitor the repositories package list for recently deleted packages, and publish a malicious version using the same name. Similarly, if an attacker can compromise the maintainer's account they can add malicious code in new versions or release malicious packages under the maintainer's name. 

Exploiting package relocation is interesting in that it does not require an attacker to compromise the maintainer's account to succeed. This attack is described as a novel attack vector by Gu et al. (2023), and is briefly described below. Certain registries do not use a web portal to manage packages and instead rely on code hosting platforms such as GitHub for maintainers to publish their software. Instead, these registries maintain a list of packages and their corresponding URLs. These hosting platforms allow users to transfer ownership to another account, which will redirect the package from the old location to the new location. However, the registry is unaware of this change and will not automatically update the location of the package. Thus, if the original account is deleted an attacker can re-register the account and cut off the automatic GitHub redirection. This results in users installing a malicious library when they attempt to install or upgrade the package from the official registry. 

\subsection{Source Code Repositories}
While artifact registries allow developers to deploy and manage software, 
source code repositories allow them to maintain and version the source code 
associated with that software. Use of these repositories is ubiquitous, allowing 
teams of developers to collaborate on all aspects of software projects, both 
closed- and open-source. More recently, automation pipelines such as GitHub 
Actions were integrated into these repositories and allow developers to automate 
processes such as testing, merging, and build. These pipelines allow users to 
configure various jobs to be run each time an event occurs, such as when code 
is pushed to the repository, and issue is closed, or a pull request is submitted.  
However, these workflows are not immune to vulnerabilities and represent 
a significant attack surface. 

Reusable actions allow developers to create a workflow and call it from 
other workflows in the same project, organization, or across GitHub. 
For example, GitHub allows developers to publish their actions on 
GitHub Marketplace which can then be used by others in a similar 
manner as software dependencies. These actions can be malicious in 
nature, or contain security vulnerabilities unintentionally included by 
their creators. A recent review of reusable actions indicated that 54\% 
were affected by at least one security weakness, with seven out of the 
top ten vulnerabilities associated with improper input validation \cite{delicheh2024mitigating}. 

Since actions allow for program execution, they can be affected by vulnerabilities in 
dependencies. JavaScript Actions are the most common type of reusable 
action \cite{delicheh2024mitigating}, which allow JavaScript to be executed in 
a Node.js environment. Many of these actions rely on npm packages, and a 
recent analysis indicated that many of these dependencies are deeply nested, 
with 91\% having indirect dependencies \cite{onsori2024quantifying}. Deep nesting 
obfuscates understanding of the imported software, and this can lead to critical 
vulnerabilities being inadvertently included in these workflows.

Aside from vulnerabilities in the workflows associated with these repositories, 
security vulnerabilities can arise from contributions from bad actors, 
especially in the context of open-source.

\subsection{Cloud Infrastructure}
Cloud computing has boomed in recent years, with companies such as Amazon, Google, 
Microsoft, and others providing cloud infrastructure to users and companies at an incredible scale. 
This service offers scalability, flexibility, and initial cost savings compared to traditional infrastructure, 
but brings new security challenges as an external party must be entrusted with the data. Given their 
role and data they have access to, it is no surprise that security is one of the primary concerns. Additionally, 
cloud infrastructure brings with it new security issues related to virtualization, data geolocation, storage, 
confidentiality, and session hijacking \cite{djenna2014security}. At a high level, the security of cloud 
infrastructure can be broken into four levels: the compute level (ex. physical server and hypervisor security), 
the network level (ex. virtual firewall and securing data in transit), the application level (ex. providing protection 
to applications utilizing the resources), and data \cite{saini2014security}. 

At the data level, cloud infrastructure faces issues such as data breaches, loss, segregation, virtualization, 
confidentiality, integrity, and availability \cite{alghofaili2021secure}. The shared nature of cloud infrastructure 
makes it a target for attackers, as success can mean stealing a significant amount of data. While there are 
many reasons for data loss and leakage, the primary reasons are due to misconfigurations, unauthorized access, insecure 
interfaces, and hijacking accounts \cite{salih2024cloud}. Attacks can be carried out by internal or external threat actors, 
and encryption and watermarking are the two major defenses to these security issues \cite{alghofaili2021secure}. 

Outside of data loss/leakage, cloud infrastructure providers face issues related to availability and integrity. Customers 
expect near constant uptime without data corruption. The distributed nature of cloud computing complicates these, 
and attackers can target availability by launching a distributed denial of service (DDoS) attack, and can attack integrity 
by corrupting any of the multiple locations the data is stored.

At the compute level, cloud infrastructure faces threats related to physical security and virtualization. To provide cloud 
infrastructure at a large scale, providers build or rent large data centers that provide the physical servers, storage, and 
networking. As these servers contain sensitive data, unauthorized access by employees or other individuals presents a 
major security threat. While physically accessing the server is infeasible for the vast majority of cybercriminals, they may 
gain access to the servers through vulnerabilities related to virtualization. Cloud computing heavily relies on virtualization, 
where many virtual machines share the same physical hardware. Vulnerabilities in the hypervisor can allow attackers to 
gain access to the physical hardware and the applications hosted by the hypervisor \cite{alghofaili2021secure}.

Cloud infrastructure is vulnerable to attacks on its network, much like other internet-based services and applications. 
Attackers can target cloud infrastructure using DDoS and domain name server (DNS) cache poisoning attacks, and packet sniffing. 
Additionally, IP addresses are commonly reassigned in cloud environments, which can increase the risk of attack if the process 
of reusing an IP address happens faster than its old assignment is removed from the DNS cache \cite{alghofaili2021secure}. 
 
Finally, cloud infrastructure faces threats at the application level such as insecure APIs and  end user attacks 
\cite{alghofaili2021secure}. Cloud computing relies on numerous APIs that connect hardware and various software components such as databases, 
authentication and authorization systems, compute engines, and others. Vulnerabilities in these APIs can provide attackers a mechanism to 
carry out attacks. Additionally, attackers may target the end users of the services in phishing schemes to attempt to gain access to their accounts. 

\subsection{Build Systems}

\section{Defending against attacks}
It is clear that there are many attack surfaces in the software supply chain, and the number of disparate components and organizations involved can make security difficult. Modern software applications are complex and commonly involve hundreds or thousands of dependencies, from software libraries to cloud infrastructure. An attacker requires just a single vulnerable component to mount an attack, and the result can affect businesses, end users, and governments. Software supply chain security has evolved in recent decades, evolving from a focus on perimeter defense, to including dependency management, software bill of materials (SBOM), zero-trust and secure build pipelines, and global standards \cite{anasuri2024software}. Additionally, new tooling has been developed to assist organizations in implementing these methods.

\subsection{Dependency Management}
As software has become more complex, managing the hundreds or thousands of dependencies has become increasingly important. At a basic level, most package managers such as npm or Maven use versioning systems that provide users with the ability to specify automatic updates. For example, users can use the "$\wedge$" operator to allow for minor or patch updates, or the "$\sim$" to allow for only patch updates. However, management of dependencies becomes much more complex when transitive dependencies are considered. On average, a user implicitly includes around 80 additional packages through transitive dependencies for each package installed by npm, and up to 40\% of packages rely on code known to be vulnerable  \cite{zimmermann2019small}. Clearly, this is a security threat to modern applications. Zimmermann et al. (2019) present some potential mitigations to this including raising developer awareness, vulnerable package warnings, code vetting, and training and vetting maintainers. 

\subsection{Units}
\begin{itemize}
\item Use either SI (MKS) or CGS as primary units. (SI units are encouraged.) English units may be used as secondary units (in parentheses). An exception would be the use of English units as identifiers in trade, such as ``3.5-inch disk drive''.
\item Avoid combining SI and CGS units, such as current in amperes and magnetic field in oersteds. This often leads to confusion because equations do not balance dimensionally. If you must use mixed units, clearly state the units for each quantity that you use in an equation.
\item Do not mix complete spellings and abbreviations of units: ``Wb/m\textsuperscript{2}'' or ``webers per square meter'', not ``webers/m\textsuperscript{2}''. Spell out units when they appear in text: ``. . . a few henries'', not ``. . . a few H''.
\item Use a zero before decimal points: ``0.25'', not ``.25''. Use ``cm\textsuperscript{3}'', not ``cc''.)
\end{itemize}

\subsection{Equations}
Number equations consecutively. To make your 
equations more compact, you may use the solidus (~/~), the exp function, or 
appropriate exponents. Italicize Roman symbols for quantities and variables, 
but not Greek symbols. Use a long dash rather than a hyphen for a minus 
sign. Punctuate equations with commas or periods when they are part of a 
sentence, as in:
\begin{equation}
a+b=\gamma\label{eq}
\end{equation}

Be sure that the 
symbols in your equation have been defined before or immediately following 
the equation. Use ``\eqref{eq}'', not ``Eq.~\eqref{eq}'' or ``equation \eqref{eq}'', except at 
the beginning of a sentence: ``Equation \eqref{eq} is . . .''

\subsection{\LaTeX-Specific Advice}

Please use ``soft'' (e.g., \verb|\eqref{Eq}|) cross references instead
of ``hard'' references (e.g., \verb|(1)|). That will make it possible
to combine sections, add equations, or change the order of figures or
citations without having to go through the file line by line.

Please don't use the \verb|{eqnarray}| equation environment. Use
\verb|{align}| or \verb|{IEEEeqnarray}| instead. The \verb|{eqnarray}|
environment leaves unsightly spaces around relation symbols.

Please note that the \verb|{subequations}| environment in {\LaTeX}
will increment the main equation counter even when there are no
equation numbers displayed. If you forget that, you might write an
article in which the equation numbers skip from (17) to (20), causing
the copy editors to wonder if you've discovered a new method of
counting.

{\BibTeX} does not work by magic. It doesn't get the bibliographic
data from thin air but from .bib files. If you use {\BibTeX} to produce a
bibliography you must send the .bib files. 

{\LaTeX} can't read your mind. If you assign the same label to a
subsubsection and a table, you might find that Table I has been cross
referenced as Table IV-B3. 

{\LaTeX} does not have precognitive abilities. If you put a
\verb|\label| command before the command that updates the counter it's
supposed to be using, the label will pick up the last counter to be
cross referenced instead. In particular, a \verb|\label| command
should not go before the caption of a figure or a table.

Do not use \verb|\nonumber| inside the \verb|{array}| environment. It
will not stop equation numbers inside \verb|{array}| (there won't be
any anyway) and it might stop a wanted equation number in the
surrounding equation.

\subsection{Some Common Mistakes}\label{SCM}
\begin{itemize}
\item The word ``data'' is plural, not singular.
\item The subscript for the permeability of vacuum $\mu_{0}$, and other common scientific constants, is zero with subscript formatting, not a lowercase letter ``o''.
\item In American English, commas, semicolons, periods, question and exclamation marks are located within quotation marks only when a complete thought or name is cited, such as a title or full quotation. When quotation marks are used, instead of a bold or italic typeface, to highlight a word or phrase, punctuation should appear outside of the quotation marks. A parenthetical phrase or statement at the end of a sentence is punctuated outside of the closing parenthesis (like this). (A parenthetical sentence is punctuated within the parentheses.)
\item A graph within a graph is an ``inset'', not an ``insert''. The word alternatively is preferred to the word ``alternately'' (unless you really mean something that alternates).
\item Do not use the word ``essentially'' to mean ``approximately'' or ``effectively''.
\item In your paper title, if the words ``that uses'' can accurately replace the word ``using'', capitalize the ``u''; if not, keep using lower-cased.
\item Be aware of the different meanings of the homophones ``affect'' and ``effect'', ``complement'' and ``compliment'', ``discreet'' and ``discrete'', ``principal'' and ``principle''.
\item Do not confuse ``imply'' and ``infer''.
\item The prefix ``non'' is not a word; it should be joined to the word it modifies, usually without a hyphen.
\item There is no period after the ``et'' in the Latin abbreviation ``et al.''.
\item The abbreviation ``i.e.'' means ``that is'', and the abbreviation ``e.g.'' means ``for example''.
\end{itemize}
An excellent style manual for science writers is.

\subsection{Authors and Affiliations}
\textbf{The class file is designed for, but not limited to, six authors.} A 
minimum of one author is required for all conference articles. Author names 
should be listed starting from left to right and then moving down to the 
next line. This is the author sequence that will be used in future citations 
and by indexing services. Names should not be listed in columns nor group by 
affiliation. Please keep your affiliations as succinct as possible (for 
example, do not differentiate among departments of the same organization).

\subsection{Identify the Headings}
Headings, or heads, are organizational devices that guide the reader through 
your paper. There are two types: component heads and text heads.

Component heads identify the different components of your paper and are not 
topically subordinate to each other. Examples include Acknowledgments and 
References and, for these, the correct style to use is ``Heading 5''. Use 
``figure caption'' for your Figure captions, and ``table head'' for your 
table title. Run-in heads, such as ``Abstract'', will require you to apply a 
style (in this case, italic) in addition to the style provided by the drop 
down menu to differentiate the head from the text.

Text heads organize the topics on a relational, hierarchical basis. For 
example, the paper title is the primary text head because all subsequent 
material relates and elaborates on this one topic. If there are two or more 
sub-topics, the next level head (uppercase Roman numerals) should be used 
and, conversely, if there are not at least two sub-topics, then no subheads 
should be introduced.

\subsection{Figures and Tables}
\paragraph{Positioning Figures and Tables} Place figures and tables at the top and 
bottom of columns. Avoid placing them in the middle of columns. Large 
figures and tables may span across both columns. Figure captions should be 
below the figures; table heads should appear above the tables. Insert 
figures and tables after they are cited in the text. Use the abbreviation 
``Fig.~\ref{fig}'', even at the beginning of a sentence.

\begin{table}[htbp]
\caption{Table Type Styles}
\begin{center}
\begin{tabular}{|c|c|c|c|}
\hline
\textbf{Table}&\multicolumn{3}{|c|}{\textbf{Table Column Head}} \\
\cline{2-4} 
\textbf{Head} & \textbf{\textit{Table column subhead}}& \textbf{\textit{Subhead}}& \textbf{\textit{Subhead}} \\
\hline
copy& More table copy$^{\mathrm{a}}$& &  \\
\hline
\multicolumn{4}{l}{$^{\mathrm{a}}$Sample of a Table footnote.}
\end{tabular}
\label{tab1}
\end{center}
\end{table}

\begin{figure}[htbp]
\centerline{\includegraphics{fig1.png}}
\caption{Example of a figure caption.}
\label{fig}
\end{figure}

Figure Labels: Use 8 point Times New Roman for Figure labels. Use words 
rather than symbols or abbreviations when writing Figure axis labels to 
avoid confusing the reader. As an example, write the quantity 
``Magnetization'', or ``Magnetization, M'', not just ``M''. If including 
units in the label, present them within parentheses. Do not label axes only 
with units. In the example, write ``Magnetization (A/m)'' or ``Magnetization 
\{A[m(1)]\}'', not just ``A/m''. Do not label axes with a ratio of 
quantities and units. For example, write ``Temperature (K)'', not 
``Temperature/K''.

\section*{Acknowledgment}

The preferred spelling of the word ``acknowledgment'' in America is without 
an ``e'' after the ``g''. Avoid the stilted expression ``one of us (R. B. 
G.) thanks $\ldots$''. Instead, try ``R. B. G. thanks$\ldots$''. Put sponsor 
acknowledgments in the unnumbered footnote on the first page.

\section*{References}
tnotes in the 
abstract or reference list. Use letters for table footnotes.


\bibliographystyle{IEEEtran}
\bibliography{references}    

\vspace{12pt}
\color{red}
IEEE conference templates contain guidance text for composing and formatting conference papers. Please ensure that all template text is removed from your conference paper prior to submission to the conference. Failure to remove the template text from your paper may result in your paper not being published.

\end{document}
