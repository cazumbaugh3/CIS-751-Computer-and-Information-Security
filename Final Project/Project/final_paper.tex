\documentclass[conference]{IEEEtran}
% \IEEEoverridecommandlockouts
% The preceding line is only needed to identify funding in the first footnote. If that is unneeded, please comment it out.
\usepackage{cite}
\usepackage{url}
\usepackage{amsmath,amssymb,amsfonts}
\usepackage{algorithmic}
\usepackage{tikz}
\usetikzlibrary{positioning}
\usepackage{graphicx}
% LaTeX Symbols: https://tug.ctan.org/info/symbols/comprehensive/symbols-a4.pdf
\usepackage{halloweenmath}
\usepackage{caption}
\captionsetup{justification=raggedright,singlelinecheck=false} % Align figure captions to the left
\usepackage{textcomp}
\usepackage{xcolor}
\def\BibTeX{{\rm B\kern-.05em{\sc i\kern-.025em b}\kern-.08em
    T\kern-.1667em\lower.7ex\hbox{E}\kern-.125emX}}
        
\begin{document}

\title{Securing the software supply chain}

\author{\IEEEauthorblockN{Charles Zumbaugh}
\IEEEauthorblockA{\textit{Dept. of Computer Science} \\
\textit{Kansas State University}\\
Manhattan, Kansas \\
cazumbaugh@ksu.edu}
}

\maketitle

\begin{abstract}
Modern software development depends heavily on layers of abstraction 
such as libraries, frameworks, cloud infrastructure, build tooling, and other 
software built and maintained by others. This speeds innovation and makes the 
development of complex systems more accessible by allowing developers to build 
upon previous work. Given it's importance to modern development, the software 
supply chain represents an important attack vector. Attackers are increasingly 
targeting this ecosystem to mount attacks, and recent incidents highlight the scale 
of the fallout when these attacks are successful. This review discusses the primary 
attack vectors in the software supply chain, security measures and frameworks to 
defend against these attacks, and recent high-profile incidents.
\end{abstract}

\begin{IEEEkeywords}
memory corruption, security, vulnerability, systems, malware
\end{IEEEkeywords}

\section{Introduction}
As computers become more embedded in everyday life, the security of these systems have become a critical concern. Cyber attacks against government and non-government organizations have increased in recent years, particularly against healthcare and large, multinational companies \cite{hammouchi2019digging}. As personal data is increasingly being collected and stored on servers, the scale of these breaches is also increasing. For example, a recent breach of Change Healthcare in 2025 exposed the personal data of more than 192 million individuals in the United States \cite{privacyrights2025}. The financial implications of security vulnerabilities are substantial, and IBM estimated the cost of a data breach in 2025 to be \$4.4 million \cite{ibm2025}.

Modern software is built upon layers of reusable abstractions such as libraries, frameworks, cloud infrastructure, and build tools that are built and maintained by third parties. This ecosystem is known as the software supply chain and it involves numerous, globally-distributed participants building software used at various phases in the development process. The 2025 Open Source Security and Risk Analysis report found that 97\% of evaluated codebases contained open source software, with 100\% of codebases in the EdTech, internet, and mobile app sectors containing open source software \cite{blackduck2025}. 

Given the massive scale of open source software and the software supply chain in general, it is not surprising that attackers are increasingly targeting this ecosystem. Sonatype reported that 34,319 new open source malware packages were identified in quarter 3 of 2025, representing a 140\% increase from quarter 2 2025 \cite{sonatype2025}. Recent attacks such as Log4j and SolarWinds highlight the importance of securing this ecosystem and the consequences of a successful attack. Aside from the economic damages associated with these attacks, distrust in the software ecosystem will undoubtedly slow technological innovation. Thus, it is essential to develop tools and frameworks to guard against malicious actors.

\section{Software Supply Chain Attack vectors}
Software supply chain attacks are considered to have three major attack vectors: dependencies, build infrastructure, and humans \cite{williams2025research}. These attack vectors span the entire software lifecycle, and thus, security risks exist in each phase of this lifecycle. This review will focus primarily on software dependencies and artifact registries, cloud infrastructure, and build systems.

\subsection{Dependencies}
Vulnerabilities and malware included in open source and third party dependencies represent a critical security threat. In many cases, unintentional vulnerabilities are included in these dependencies. While such vulnerabilities are generally discovered, patched, and made public, developers and administrators may be slow to patch. For example, many systems were still vulnerable to attacks such as Heartbleed and Shellshock after the patch was released, in some cases years after \cite{10154229}. A recent analysis of major package managers found that technical lag is common, with the majority of fixed version declarations being outdated and a significant number of flexible version declarations being outdated \cite{9359290}. While there are likely many reasons for this, research has shown a strong presence of technical lag caused by the use of dependency constraints in the JavaScript package manager, npm, suggesting that developers may be reluctant to update dependencies due to backwards compatibility \cite{zerouali2018empirical}. This idea is supported by qualitative research which demonstrated that developers are more likely to adopt a security fix \textit{not} bundled with functionality improvements because they are less likely to introduce breaking changes \cite{pashchenko2020qualitative}. Maintaining software ecosystems is costly, especially if updates are frequent \cite{berhe2023maintenance}, but the technical lag generated by missing updates can result in an increased risk of critical vulnerabilities.

Attackers may also attempt to infect programs by typosquatting, the practice of uploading a package with a similar name as that of a very popular package (ex. reacte vs react). If a developer makes a typo while installing the package (ex. \textit{npm install reacte}), the malicious software is installed, which can target either the developer's or end user's machine. This practice is common in software libraries and other areas of the internet. While the effectiveness at tricking users varies based on how the fake name is composed. Users are more likely to correctly identify typosquatting that adds characters to the name, and are more likely to be deceived by permutations and substitutions of characters \cite{spaulding2017understanding}. Due to its prevalence, several software tools have been developed in recent years to guard against these attacks. A recently developed tool for identifying and reporting typosquatted imports, TypoGard, was able to flag up to 99.4\% of known typosquatted imports on npm, PyPI, and RubyGems \cite{taylor2020defending}. At a high level TypoGard compares the package name against a list of popular packages, and flags it if the name matches after transformations. These tools also exist to protect against typosquatted URLs online with good results. For example, TypoWriter uses a recurrent neural network to identify probable typo variations during the pre-registration phase \cite{ahmad2019typowriter}, and TypoAlert is a Chrome extension that can alert users to alert users when they attempt to visit a typosquatted website \cite{blefari2024typoalert}.

\subsection{Artifact Registries}
An artifact registry or repository is a centralized system that stores, manages, and versions software artifacts such as libraries, compiled code, and container images. These registries provide a mechanism for package developers to deploy and maintain their software, and allow developers to easily consume these packages using package managers. Attacks on artifact registries are closely related to dependency vulnerabilities, as they commonly rely on developers unknowingly installing malicious software. Attackers attempt to hijack legitimate packages in these registries by attacking deleted packages, compromising the maintainer's account, or by exploiting package relocation in decentralized registries \cite{gu2023investigating}. In the first case, attackers simply monitor the repositories package list for recently deleted packages, and publish a malicious version using the same name. Similarly, if an attacker can compromise the maintainer's account they can add malicious code in new versions or release malicious packages under the maintainer's name. 

Exploiting package relocation is interesting in that it does not require an attacker to compromise the maintainer's account to succeed. This attack is described as a novel attack vector by Gu et al. (2023), and is briefly described below. Certain registries do not use a web portal to manage packages and instead rely on code hosting platforms such as GitHub for maintainers to publish their software. Instead, these registries maintain a list of packages and their corresponding URLs. These hosting platforms allow users to transfer ownership to another account, which will redirect the package from the old location to the new location. However, the registry is unaware of this change and will not automatically update the location of the package. Thus, if the original account is deleted an attacker can re-register the account and cut off the automatic GitHub redirection. This results in users installing a malicious library when they attempt to install or upgrade the package from the official registry. 

\subsection{Source Code Repositories}
While artifact registries allow developers to deploy and manage software, 
source code repositories allow them to maintain and version the source code 
associated with that software. Use of these repositories is ubiquitous, allowing 
teams of developers to collaborate on all aspects of software projects, both 
closed- and open-source. More recently, automation pipelines such as GitHub 
Actions were integrated into these repositories and allow developers to automate 
processes such as testing, merging, and build. These pipelines allow users to 
configure various jobs to be run each time an event occurs, such as when code 
is pushed to the repository, and issue is closed, or a pull request is submitted.  
However, these workflows are not immune to vulnerabilities and represent 
a significant attack surface. 

Reusable actions allow developers to create a workflow and call it from 
other workflows in the same project, organization, or across GitHub. 
For example, GitHub allows developers to publish their actions on 
GitHub Marketplace which can then be used by others in a similar 
manner as software dependencies. These actions can be malicious in 
nature, or contain security vulnerabilities unintentionally included by 
their creators. A recent review of reusable actions indicated that 54\% 
were affected by at least one security weakness, with seven out of the 
top ten vulnerabilities associated with improper input validation \cite{delicheh2024mitigating}. 

Since actions allow for program execution, they can be affected by vulnerabilities in 
dependencies. JavaScript Actions are the most common type of reusable 
action \cite{delicheh2024mitigating}, which allow JavaScript to be executed in 
a Node.js environment. Many of these actions rely on npm packages, and a 
recent analysis indicated that many of these dependencies are deeply nested, 
with 91\% having indirect dependencies \cite{onsori2024quantifying}. Deep nesting 
obfuscates understanding of the imported software, and this can lead to critical 
vulnerabilities being inadvertently included in these workflows.

Aside from vulnerabilities in the workflows associated with these repositories, 
security vulnerabilities can arise from contributions from bad actors, 
especially in the context of open-source.

\subsection{Cloud Infrastructure}
Cloud computing has boomed in recent years, with companies such as Amazon, Google, 
Microsoft, and others providing cloud infrastructure to users and companies at an incredible scale. 
This service offers scalability, flexibility, and initial cost savings compared to traditional infrastructure, 
but brings new security challenges as an external party must be entrusted with the data. Given their 
role and data they have access to, it is no surprise that security is one of the primary concerns. Additionally, 
cloud infrastructure brings with it new security issues related to virtualization, data geolocation, storage, 
confidentiality, and session hijacking \cite{djenna2014security}. At a high level, the security of cloud 
infrastructure can be broken into four levels: the compute level (ex. physical server and hypervisor security), 
the network level (ex. virtual firewall and securing data in transit), the application level (ex. providing protection 
to applications utilizing the resources), and data \cite{saini2014security}. 

At the data level, cloud infrastructure faces issues such as data breaches, loss, segregation, virtualization, 
confidentiality, integrity, and availability \cite{alghofaili2021secure}. The shared nature of cloud infrastructure 
makes it a target for attackers, as success can mean stealing a significant amount of data. While there are 
many reasons for data loss and leakage, the primary reasons are due to misconfigurations, unauthorized access, insecure 
interfaces, and hijacking accounts \cite{salih2024cloud}. Attacks can be carried out by internal or external threat actors, 
and encryption and watermarking are the two major defenses to these security issues \cite{alghofaili2021secure}. 

Outside of data loss/leakage, cloud infrastructure providers face issues related to availability and integrity. Customers 
expect near constant uptime without data corruption. The distributed nature of cloud computing complicates these, 
and attackers can target availability by launching a distributed denial of service (DDoS) attack, and can attack integrity 
by corrupting any of the multiple locations the data is stored.

At the compute level, cloud infrastructure faces threats related to physical security and virtualization. To provide cloud 
infrastructure at a large scale, providers build or rent large data centers that provide the physical servers, storage, and 
networking. As these servers contain sensitive data, unauthorized access by employees or other individuals presents a 
major security threat. While physically accessing the server is infeasible for the vast majority of cybercriminals, they may 
gain access to the servers through vulnerabilities related to virtualization. Cloud computing heavily relies on virtualization, 
where many virtual machines share the same physical hardware. Vulnerabilities in the hypervisor can allow attackers to 
gain access to the physical hardware and the applications hosted by the hypervisor \cite{alghofaili2021secure}.

Cloud infrastructure is vulnerable to attacks on its network, much like other internet-based services and applications. 
Attackers can target cloud infrastructure using DDoS and domain name server (DNS) cache poisoning attacks, and packet sniffing. 
Additionally, IP addresses are commonly reassigned in cloud environments, which can increase the risk of attack if the process 
of reusing an IP address happens faster than its old assignment is removed from the DNS cache \cite{alghofaili2021secure}. 
 
Finally, cloud infrastructure faces threats at the application level such as insecure APIs and  end user attacks 
\cite{alghofaili2021secure}. Cloud computing relies on numerous APIs that connect hardware and various software components such as databases, 
authentication and authorization systems, compute engines, and others. Vulnerabilities in these APIs can provide attackers a mechanism to 
carry out attacks. Additionally, attackers may target the end users of the services in phishing schemes to attempt to gain access to their accounts. 

\section{Defending against attacks}
It is clear that there are many attack surfaces in the software supply chain, and the number of disparate components and organizations involved can make security difficult. Modern software applications are complex and commonly involve hundreds or thousands of dependencies, from software libraries to cloud infrastructure. An attacker requires just a single vulnerable component to mount an attack, and the result can affect businesses, end users, and governments. Software supply chain security has evolved in recent decades, evolving from a focus on perimeter defense, to including dependency management, software bill of materials (SBOM), zero-trust and secure build pipelines, and global standards \cite{anasuri2024software}. Additionally, new tooling has been developed to assist organizations in implementing these methods.

\subsection{Dependency Management}
As software has become more complex, managing the hundreds or thousands of dependencies has become increasingly important. At a basic level, most package managers such as npm or Maven use versioning systems that provide users with the ability to specify automatic updates. For example, users can use the "$\wedge$" operator to allow for minor or patch updates, or the "$\sim$" to allow for only patch updates. However, management of dependencies becomes much more complex when transitive dependencies are considered. On average, a user implicitly includes around 80 additional packages through transitive dependencies for each package installed by npm, and up to 40\% of packages rely on code known to be vulnerable  \cite{zimmermann2019small}. Clearly, this is a security threat to modern applications. Zimmermann et al. (2019) present some potential mitigations to this including raising developer awareness, vulnerable package warnings, code vetting, and training and vetting maintainers. Most modern package managers provide warnings for vulnerable packages, but the vulnerability must be known and they generally do not address vulnerabilities in transitive dependencies. While manual code vetting is likely infeasible at scale, automated analyzers could be adopted by package registries to analyze each update \cite{zimmermann2019small}. Similar processes are already in use in other areas. For example, the Apple App Store performs static analysis on all submitted binaries to help maintain the security and privacy of the Apple ecosystem.

\subsection{Software Bill of Materials}
A SBOM is a formal record containing the details and supply chain relationships of various components used in building software \cite{ntia2021minimum}. The National Telecommunications and Information Administration (NTIA) provides some minimum elements for a SBOM, shown in Table \ref{sbom}. 

\begin{table}[htbp]
\caption{Minimum elements of a SBOM as defined by NTIA \cite{ntia2021minimum}}
\label{sbom}
\begin{tabular}{|c|}
\hline
\textbf{Item} \\
\hline
Supplier \\
\hline
Component name \\
\hline
Version of component \\
\hline
Unique identifiers \\
\hline
Dependency relationship \\
\hline
Author of SBOM \\
\hline
Timestamp \\
\hline
Provide support for SBOM automation \\
\hline
Define the operations of SBOM requests, generation, and use\\
\hline
\end{tabular}
\end{table}

The SBOM is intended to provide developers and users of software with information regarding the supply chain, allowing them to track known and recently discovered security vulnerabilities. Recent high profile supply chain cyber attacks, such as Log4Shell, have highlighted the need for SBOMs, as they allowed companies to very quickly patch the vulnerability \cite{rl2021log4shell}. There are three primary formats for SBOMs: 1) Software Package Data eXchange (SPDX), 2) CycloneDX, and 3) Software Identification (SWID) Tagging \cite{xia2023empirical}. In 2021, the White House issues an executive order requiring all companies conducting business with the US government to provide SBOMs \cite{house2021executive}. However, despite these mandates, adoption in the industry and open source community is lagging. A large proportion of widely used software do not have SBOMs, and there is a lack of consensus regarding what to include in SBOMs, despite the official NTIA recommendations \cite{xia2023empirical}. The lack of widespread adoption is likely due, in part, to challenges with automated SBOM tools, lack of interoperability, time consumption, and risks of transparency \cite{zahan2023software}.

\subsection{Zero-trust build pipelines}
Zero trust architecture is a security paradigm that is based on the premise that trust is never implicitly given and must be continually evaluated \cite{stafford2020zero}. In the context of cloud-native environments, this means that every component, artifact, and action must be continuously verified, and all pipeline components from source code to build artifacts are treated as compromised until proven otherwise \cite{bondalapati2025secure}. Zero trust employs several principles, such as identity verification and access management, micro-segmentation, and constant observation.

Organizations can employ multi-factor authentication (MFA) at various stages of the DevOps pipeline, and demand two or more verification factors - something they know, something they have, or something they are. Additionally, incorporating attribute-based access control account factors such as device location, device kind, and access duration can be used with MFA to decrease the potential harm caused by compromised credentials, and reduces the danger of unauthorized access to the build pipeline \cite{vemula2022integrating}. In cloud AaaS environments, this is commonly handled through the identity and access management (IAM) policies. These platforms enable end users to enforce the zero-trust model by enabling MFA, using identity federation for consistent authentication, enforcing role-based access controls, and regularly auditing privileges \cite{vermazero}.

At the DevOps level, dividing the build pipeline into smaller, distinct pipeline phases can reduce the attack surface and may mitigate some of the consequences of a security breach. Isolating the development, testing, and production environments can improve security if strict access controls are put in place to determine which individuals can access each phase \cite{vemula2022integrating}. Additionally, linking security tools within the CI/CD pipeline, implementing automated policy enforcement, and adopting security-as-code can help to detect vulnerabilities and misconfigurations prior to deployment \cite{vermazero}.

Constant observation and logging allow for organizations to quickly identify breaches and monitor for malicious activity. Threat detection and continuous monitoring can be integrated in the build pipeline by integrating security information and event management systems, using machine learning for anomaly detection, and automating incident responses \cite{vemula2022integrating}. Constant observation and automated incident responses can reduce the mean time to respond by 59\% and improve resolution accuracy by 73\% \cite{bondalapati2025secure}. Various tools have been developed to address this, such as Prometheus \cite{prometheus} and Falco \cite{falco}, which can integrate into cloud environments such as Google Cloud and Amazon AWS.

\section{Recent software supply chain security incidents}
In recent years, several high-profile supply chain attacks have occurred that highlight the critical need for security at all levels of the chain. One of these attacks, the SolarWinds breach, prompted the White House to respond with Executive Order 14028, an order mandating cybersecurity enhancements for federal agencies and partners and instating the SBOM requirement for government contractors. 

\subsection{Log4Shell}
The Log4Shell vulnerability (CVE-2021-44228) was a zero-day vulnerability reported in November 2021 in Apache Log4j that allows attackers to execute arbitrary code \cite{log4shellcve}.

\bibliographystyle{IEEEtran}
\bibliography{references}    

\end{document}
